\documentclass[12pt]{article}
\usepackage{fullpage}
\usepackage{amsmath,amsopn}
\usepackage{graphicx}
\usepackage{color}
\usepackage{verbatim}
\usepackage{setspace}
\usepackage{gensymb}
\usepackage{float}
\usepackage[normalem]{ulem}
%\usepackage[dvips,bookmarks=false,colorlinks,urlcolor=blue,pdftitle={
\usepackage{caption}
\usepackage{subcaption}
\usepackage{amsmath}
\usepackage{amssymb}
\usepackage{amsfonts}
\usepackage{amsthm}
\usepackage{comment}

\def\vO{{\bf O}}
\def\vP{{\bf P}}
\def\vp{{\bf p}}
\def\vx{{\bf x}}
\def\vl{{\bf l}}
\def\mS{{\bf S}}
\def\mT{{\bf T}}
\def\mH{{\bf H}}
\def\mA{{\bf A}}
\def\tx{{\tilde{\bf x}}}
\def\ta{{\tilde{\bf a}}}
\def\tb{{\tilde{\bf b}}}
\def\tc{{\tilde{\bf c}}}
\def\hn{{\bf \hat{n}}}
\def\hv{{\bf \hat{v}}}
\def\hh{{\bf \hat{h}}}
\def\vh{{\bf h}}
\def\vs{{\bf s}}
\def\hs{{\bf \hat{s}}}
\newcommand{\R}{\mathbb{R}}
\newcommand{\ud}{\,\mathrm{d}}

\usepackage[bookmarks=false,colorlinks,urlcolor=blue,pdftitle={Take-home Quiz 1},pdfauthor={Matthew O'Toole}]{hyperref}

\begin{document}

\begin{centering}
	{\large 16-385 Computer Vision, Fall 2020\\}
	\vspace{2ex}
	{\LARGE Take-home Quiz 1\\}
	\vspace{2ex}
	{\large Due Date: Monday September 21, 2020 23:59 ET\\}
\end{centering}

\bigskip

\section*{Question 1 \emph{\small(5 points)}}

The continuous convolution of two functions
$f\left(x\right)$ and $g\left(x\right)$ is given by
\begin{equation}
	\left(f \ast g \right)\left(x\right) = \int_{-\infty}^{+\infty} f\left(y\right)
g\left(x - y\right) \ud y. \label{eq:continuous:convolution}
\end{equation}
The Gaussian function at scale $s$ is defined as
\begin{equation}
	G_s \left(x\right) = \frac{1}{\sqrt{2 \pi s}} \exp\left(-\frac{x^2}{2s}\right),
\label{eq:gaussian}
\end{equation}
and has the property that
\begin{equation}
	\int_{-\infty}^{+\infty} G_s\left(x\right) \ud x = 1. \label{eq:norm}
\end{equation}
Prove that this class of functions satisfies the \emph{semigroup property}: the
convolution of one Gaussian with another produces a third Gaussian with scale equal to
their sum, or
\begin{equation}
	\left(G_{s_1} \ast G_{s_2}\right) \left(x\right) = G_{s_1 + s_2} \left(x\right).
\label{eq:semi:group}
\end{equation}
  
\section*{Question 2 \emph{\small(5 points)}}

In lecture, we briefly discussed the convolution theorem, which states that the Fourier transform of the convolution of two functions is the product of their Fourier transforms:
\begin{equation}
\mathcal{F}\{f * g\} = \mathcal{F}\{f\}\mathcal{F}\{g\} ,
\end{equation}
where
\begin{equation}
\mathcal{F}\{f\}(k) = \int_{-\infty}^{\infty} f(x) e^{-i2\pi x k} dx
\label{eq:fourier}
\end{equation}
is the Fourier transform of a continuous function $f(x)$.  Using the definitions of the convolution operation (Equation~\eqref{eq:continuous:convolution}) and Fourier transform (Equation~\eqref{eq:fourier}), prove that the convolution theorem holds true.


\section*{Instructions}

\begin{enumerate}
    \item \textbf{Integrity and collaboration:} Students are encouraged to work
    in groups but each student must submit their own work. If you work as a
    group, include the names of your collaborators in your write up. Plagiarism
    is strongly prohibited and may lead to failure of this course.

    \item \textbf{Questions:} If you have any questions, please look at Piazza
    first. Other students may have encountered the same problem, and it may be solved
    already. If not, post your question on the discussion board. Teaching staff will
    respond as soon as possible.

    \item \textbf{Write-up:} Your write-up should be typeset in \LaTeX and should consist of your answers to the theory questions. Please note that we
    \textbf{\color{red}do not} accept handwritten scans for your write-up in quizzes.

    \item \textbf{Submission:} Your submission for this take-home quiz should be a
    PDF file, \texttt{<andrew-id.pdf>}, with your write-up. \textbf{\color{red}Please do not submit ZIP files.}
\end{enumerate}

\end{document}

